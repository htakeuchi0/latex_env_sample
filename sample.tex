\documentclass[twocolumn, uplatex, dvipdfmx]{jsarticle}

% タイトル,著者,日付
\title{タイトル}
\author{著者}
\date{日付}
 
% 使用パッケージの読み込み
\usepackage{sample}

% 参照確認用
% \usepackage{refcheck}

% 索引に必要
\makeindex

% 本文は日本語で記述する.
% 句読点として原則全角カンマ+全角ピリオドを使用するが,
% 直前が英数字や数式の場合は,半角カンマや半角ピリオドを使うこともある.
% 句点,読点は使用しない.
\begin{document}

% タイトル,著者,日付の出力
\maketitle

\begin{abstract}
	% 索引用のindexコマンドは,\index{よみ@索引語}と記述する.よみは省略できる.
	% 別の索引後にリンクさせたい場合は \index{よみ@索引語|see{リンク先索引語}} とする.
	本稿は\LaTeX\index{latex@\LaTeX}のサンプルファイルである.
\end{abstract}

\section{サンプル}

以下の式,
\begin{align}
	e^{iz}=\cos z+i\sin z,\quad z\in\mathbb{C}\label{euler1}
\end{align}
は,オイラーの公式 (Euler's formula) %
\index{おいらーのこうしき@オイラーの公式}\index{Euler's formula|see{オイラーの公式}}%
と呼ばれる\cite{sugiura:1980:1}.
式\eqref{euler1}において $z=\pi$ とおくことで得られる等式 $e^{i\pi}+1=0$ はオイラーの等式 (Euler's identity) %
\index{おいらーのとうしき@オイラーの等式}\index{Euler's identity|see{オイラーの等式}}%
と呼ばれる\cite{wiki:eulersidentity}.

% 参考文献.引用タグは "筆頭著者:発行年:同一発行年文献の区別用記号" など.
% インターネット上の文献の場合は規定しない.
\begin{thebibliography}{9}
	\bibitem{sugiura:1980:1} 杉浦 光夫,基礎数学2 解析入門I, 東京大学出版会,東京,1980.
	\bibitem{wiki:eulersidentity} Wikipedia, Euler's identity - Wikipedia, \\\url{https://en.wikipedia.org/wiki/Euler%27s_identity}, 最終閲覧2019年12月14日.
\end{thebibliography}

% 索引の出力
\printindex

\end{document}
